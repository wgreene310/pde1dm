\documentclass{article}
\usepackage{amsmath}
%\usepackage[pdftex]{hyperref} 
\usepackage{hyperref} 
%\usepackage[pdftex]{graphicx}
\usepackage{graphicx}
\usepackage{listings}
\usepackage{verbatim}
\usepackage[dvipsnames]{xcolor}
\usepackage{graphicx}
%\usepackage{svg}

\addtolength{\hoffset}{-0.5in}
\addtolength{\textwidth}{1in}
\addtolength{\voffset}{-0.5in}
\addtolength{\textheight}{1in}

\begin{document}

\newcommand{\pde}{ {\tt pde1dm} }
\newcommand{\ml}{ {\tt MATLAB} }
\newcommand{\pdepe}{ {\tt pdepe} }
\newcommand{\exdir}{c:/wgreene/scripts/matlab/pde1dM/documents/}
\newcommand{\mylisting}[1]{\lstinputlisting{\exdir#1}}
\newcommand{\pux}{\frac{\partial u}{\partial x}}
\newcommand{\mycode}[1]{{\tt #1}}

%defaults for listing command
\lstset{
    language=matlab,
    basicstyle=\ttfamily,columns=fullflexible,
    keywordstyle=\color{black},%
    identifierstyle=\color{black},%
    stringstyle=\color{Peach},
    commentstyle=\color{Green}
}

\title{Solving Partial Differential Equations With the \pde Function and MATLAB/Octave}
\author{Bill Greene}
\date{Version 1.4, July 06, 2023}
\maketitle

\tableofcontents
\newpage

\section{Overview}
The \pde function works in either MATLAB or Octave and solves systems of partial differential equations (PDE) and, optionally,
coupled ordinary differential equations (ODE) of the following form:

\begin{equation}\label{eq:pde}
c(x, t,u,\pux{})\frac{\partial u}{\partial t} = x^{-m} \frac{\partial}{\partial x}
\left(x^m f(x, t,u,\pux )\right) +  s(x, t,u,\pux ) 
\end{equation}

\begin{equation}\label{eq:ode}
F(t,v,\dot v,\tilde x, \tilde u, \partial \tilde u/\partial x, \tilde f,  \partial\tilde u/\partial t,  \partial^2\tilde u/\partial x\partial t) = 0
\end{equation}

where

\begin{tabular}{lp{4in}}
 $x$ &  Independent spatial variable. \\
$t$  &  Independent variable, time. \\
 $u$ & Vector of dependent variables defined at every spatial location. \\
$c$ & Vector of diagonal entries of the so-called ``mass matrix''. Note
the entries can be functions of $x, t, u,$ and $du/dx$.  \\
$f$ & Vector of flux entries. Note
the entries can be functions of $x, t, u,$ and $du/dx$.  \\
$s$ & Vector of source entries. Note
the entries can be functions of $x, t, u,$ and $du/dx$.  \\
m & Allows for problems with spatial cylindrical or spherical symmetry. If $m=0$,
no symmetry is assumed, i.e. a basic Cartesian coordinate system. If $m=1$,
cylindrical symmetry is assumed. If $m=2$, spherical symmetry is assumed. \\
$F$ & Vector defining a system of ODE in so-called implicit form. That is, a function of
the indicated variables that must equal zero for a solution. \\
$v$ & Vector of dependent ODE variables. \\
$\tilde x$ & Vector of spatial locations where the ODE system is coupled to the PDE system.
That is, the PDE variables are evaluated at these specific values of $x$ so they can be used
in defining the system of ODE. This set of PDE variables is denoted $\tilde u$.
\end{tabular}

\medskip
Equation \eqref{eq:pde} is defined on the interval $a\le x \ge b$.
The number of PDE in the problem will be denoted as $N$ and the number of ODE (which may be zero) will be
denoted as $M$.

The complete definition of the problem to be solved includes initial conditions (values of the solution variables at the initial time) and,
for the PDE variables, boundary conditions. The boundary conditions are defined at the ends of the spatial domain-- $x=a$ and $x=b$.
These boundary conditions take the following form
\begin{equation}\label{eq:bc}
p(x,t,u) + q(x,t)f(x,t,u,du/dx) = 0
\end{equation}
where $p$ and $q$ must be defined by the user at both ends of the domain.


\section{Calling \pde}
The basic calling sequence for \pde when there are no ODE is
\begin{lstlisting}
 solution = pde1dm(m,pdeFunc,icFunc,bcFunc,meshPts,timePts)
\end{lstlisting}
\begin{tabular}{lp{4in}}
\mycode{m} & Defines the spatial coordinate system type, as described above. \\
\mycode{pdeFunc} & Handle to a user-written function that describes the system of PDE to be solved.
This function is described in detail below. \\
\mycode{icFunc} & Handle to a user-written function that describes the initial conditions for the system of PDE.
This function is described in detail below. \\
\mycode{bcFunc} & Handle to a user-written function that describes the boundary conditions for the system of PDE.
This function is described in detail below. \\
\mycode{meshPts} & Vector of x locations defining the spatial mesh. 
The first entry in \mycode{meshPts} must equal the beginning of the interval, $a$, the last point must equal $b$, and
the values of the intermediate points must be monotonically increasing.
The accuracy of the solution depends
on the density of the points in this mesh. The spacing between points need not be uniform; it is often advantageous to
prescribe a higher density of points in places where the solution is changing rapidly.\\
\mycode{timePts} & Vector of time points where it is desired to output the solution. The density of points in this
vector has no effect on the accuracy of solution. Often the number of points is determined by the number required to
produce a smooth plot of the solution as a function of time. \\
\mycode{solution} & Values of the PDE variables at each time and mesh point. The 
size of this output matrix is number of time points $\times$ number of mesh points
$\times$ N. 
\\[.1in]
\end{tabular}

An additional argument, \mycode{options}, may be included to change various parameters
controlling the behavior of \pde.
\begin{lstlisting}
 solution = pde1dm(m,pdeFunc,icFunc,bcFunc,meshPts,timePts,options)
\end{lstlisting}
This argument is a structure array. The name, default value, and purpose of the supported options are 
\par
\begin{tabular}{llp{4in}}
\\[.1in]
Name & Default & Purpose \\
\hline
RelTol&1e-3 & Relative tolerance for converged solution \\
AbsTol&1e-6 & Absolute tolerance for converged solution \\
 Vectorized&false& If set to true, \mycode{pdeFunc} is called with a vector
 of x-values and is expected to return values of c, f, and s
 for all of these x-values. Setting this option to \mycode{true}
 substantially improves performance.
\\[.1in]
\end{tabular}

These function signatures are identical to the \ml \pdepe function.

When ODE are included in the problem definition, three additional arguments to the
\pde function are required.
\begin{lstlisting}
 [solution,odeSolution] = pde1dm(m,pdeFunc,icFunc,bcFunc,meshPts,timePts,...
                           odeFunc, odeIcFunc,xOde)
\end{lstlisting}
\begin{tabular}{lp{4in}}
\mycode{odeFunc} & Handle to a user-written function that describes the system of ODE to be solved.
This function is described in detail below. \\
\mycode{odeIcFunc} &Handle to a user-written function that describes the initial conditions for the system of ODE.
This function is described in detail below. \\
\mycode{xOde} & Vector of x locations where the systems of PDE and ODE interact. At these x locations, the values of various PDE variables
are made available for use in defining the system of ODE. Any reasonable number of x locations may be defined and they need not coincide with
the mesh points in the PDE definition. \\
\mycode{odeSolution} & Matrix of the values of ODE variables at the time points.
The size of this output matrix is M $\times$ number of time points.
\\[.1in]
\end{tabular}

The same  {\tt options} argument, described above, may also be included when there are
ODE.
\begin{lstlisting}
 [solution,odeSolution] = pde1dm(m,pdeFunc,icFunc,bcFunc,meshPts,timePts,...
                           odeFunc, odeIcFunc,xOde,options)
\end{lstlisting}

%
\section{User-Defined Functions}
The user-written functions, mentioned above, that define the systems of PDE and ODE are discussed in more detail in this section.
In the description of
these functions, a function name is chosen which is somewhat descriptive of the function's purpose; however the
user is free to choose any allowable function name for these functions.

\subsection{User-Defined Functions When The Problem Has Only PDE}

\subsubsection{PDE Definition Function}
\begin{lstlisting}
[c, f, s] = pdeFunc(x, t, u, DuDx)
\end{lstlisting}
The user must return the $c$, $f$, and $s$ matrices defined in equation \eqref{eq:pde}. The size of these matrices is N $\times$ number of entries in
the x vector. 

\subsubsection{PDE Boundary Condition Function}
\begin{lstlisting}
[pLeft, qLeft, pRight, qRight] = bcFunc(xLeft, uLeft, xRight, uRight, t)
\end{lstlisting}
The user must return the $q$ and $p$ vectors defined in equation \eqref{eq:bc} at the \mycode{Left}
and \mycode{Right} ends of the domain. The input variables, \mycode{xLeft, uLeft, xRight, uRight, t}
may be used in defining \mycode{pLeft, qLeft, pRight, qRight}. (Note that \mycode{xLeft} will
equal $a$ and \mycode{xRight} will equal $b$).
       Entries in the  \mycode{qLeft} and   \mycode{qRight}  vectors that are zero at the initial time, are assumed to be zero
       at all future times. Entries in the \mycode{qLeft} and   \mycode{qRight}  vectors that are non-zero at the initial time,
       are assumed to be non-zero at all future times.


\subsubsection{PDE Initial Condition Function}
\begin{lstlisting}
u0=pdeIcFunc(x)
\end{lstlisting}
 The complete specification of a PDE system requires that initial conditions
 (solution at the initial time) be defined by the user. The function must return the 
 initial condition,  \mycode{u0}, at spatial location  \mycode{x}.  \mycode{u0} is a vector which has
length $N$. Formally,
 the boundary conditions returned from  \mycode{bcFunc} and the initial conditions
 returned from  \mycode{pdeIcFunc} should agree but this is not strictly required by
 \pde.

\subsection{User-Defined Functions When The Problem Has Both PDE and ODE}
The following two functions are required only if the problem definition includes ODE.
\subsubsection{ODE Definition Function}
\begin{lstlisting}
F=odeFunc(t,v,vdot,x,u,DuDx,f, dudt, du2dxdt)
\end{lstlisting}
 The system of ODE is defined by equation \eqref{eq:ode}.
The input variables to \mycode{odeFunc} are
\par
\begin{tabular}{lp{4in}}
\mycode{t} & time \\
\mycode{v} & vector of ODE values \\
\mycode{vDot} & derivative of ODE variables with respect to time \\
\mycode{x} & Vector of spatial locations where the ODE couple with the PDE variables. This is referred to as $\tilde x$ in equation  \eqref{eq:ode}.\\
\mycode{u} & values of the PDE variables at the x locations \\
\mycode{DuDx} & derivatives of the PDE variables with respect to x, 
evaluated at the x locations \\
 \mycode{f} & values of the flux defined by the PDE definition evaluated at the x locations \\
\mycode{dudt} & derivatives of the PDE variables with respect to time, evaluated at
        the x locations \\
\mycode{du2dxdt} & second derivatives of the PDE variables with respect to 
           x and time, evaluated at the x locations
\\[.1in]
\end{tabular}
The vector \mycode{F} must be returned.
\subsubsection{ODE Initial Condition Function}
\begin{lstlisting}
v0=odeIcFunc()
\end{lstlisting}
A vector (length M) of initial values of the ODE variables, \mycode{v0}, must be returned.

\subsubsection{PDE Definition Function}
\begin{lstlisting}
[c, f, s] = pdeFunc(x, t, u, DuDx, v, vDot)
\end{lstlisting}
The PDE function, \mycode{pdeFunc}, is identical to the definition
above except that, when ODE are included, two additional input variables
are provided. These are the values of the ODE variables, \mycode{v} and their
derivatives with respect to time, \mycode{vDot}.

\subsubsection{PDE Boundary Condition Function}
\begin{lstlisting}
[pLeft, qLeft, pRight, qRight] = bcFunc(xLeft, uLeft, xRight, uRight, 
                                            t,v,vDot)                                       
\end{lstlisting}
The boundary condition function, \mycode{bcFunc}, is 
identical to the definition
above except that, when ODE are included, two additional input variables
are provided. These are the values of the ODE variables, \mycode{v} and their
derivatives with respect to time, \mycode{vDot}.

\section{Examples}

\subsection{Heat Conduction in a Rod}
The first example we will consider is the heat conduction in a rod where the
temperature at the left end is prescribed as $100\ ^{\circ}C$ and all other
surfaces of the rod are insulated. The initial temperature of all other points
in the rod is zero. The material is copper with density,  $\rho=8940\ kg/m^3$;
specific heat, $c_p=390\ J/(kg\ ^{\circ}C)$ and thermal conductivity, $k=385\ W/(m\ ^{\circ}C)$.

The PDE describing this behavior is 

\begin{equation}\label{eq:heat_pde}
	\rho c_p \frac{\partial T}{\partial t} = 
	\frac{\partial}{\partial x}\left(k\frac{\partial T}{\partial x}\right)
\end{equation}
With appropriate changes in coefficients, this equation describes a wide
variety of physical behavior so is particularly appropriate as a first example.

\subsubsection{Code Required by \pde}
The complete code for this example is shown below. This section highlights 
some key pieces of this code.

\begin{lstlisting}
n=12; % number of nodes in x
L=1; % length of the bar in m
x = linspace(0,L,n);
\end{lstlisting}
This code snippet defines the x-locations of the nodes. This example uses
twelve nodes but since the accuracy of the solution depends on the mesh it
is important to verify that the mesh is sufficiently refined. 

\begin{lstlisting}
t = linspace(0,2000,30); % number of time points for output of results
\end{lstlisting}
This code snippet defines the points in time where we would like to save the
solution. The accuracy of the solution is not affected by this choice.

The PDE, equation \eqref{eq:heat_pde}, is defined by matching the terms with
those in the general equation \eqref{eq:pde}. The variable $m$ is zero because
we have a rectangular Cartesian coordinate system. The resulting code is
\begin{lstlisting}
function [c,f,s] = heatpde(x,t,u,DuDx)
rho=8940; % material density
cp=390; % material specific heat
k=385; % material thermal conductivity 
c = rho*cp;
f = k*DuDx;
s = 0;
end
\end{lstlisting}
The actual name for this function is arbitrary. 
The boundary conditions are defined at each end to match equation \eqref{eq:bc}. 
At the left end, we are simply prescribing the value of temperature so $q=0$ and
$p$ is defined as the temperature at the left end minus the prescribed value of
temperature, $Tl-T0$. A common mistake is to simply set $p=T0$ instead of in
the form required by \eqref{eq:bc}, $Tl-T0$! The right end is insulated so the
heat flux, $k\partial T/\partial x$, is zero. Note carefully that we have defined
the $f$ result in the \mycode{heatpde} function as the heat flux. We want \pde to
maintain this quantity as zero at the right end so, referring to equation \eqref{eq:bc}, $p=0$ and $q=1$. The resulting code is
\begin{lstlisting}
function [pl,ql,pr,qr] = heatbc(xl,Tl,xr,Tr,t)
T0=100; % temperature at left end, degrees C
pl = Tl-T0;
ql = 0;
pr = 0;
qr = 1;
end
\end{lstlisting}
Defining the initial conditions is straightforward. We want the initial temperature
of the rod to be zero everywhere except at the left end where it should match the
boundary condition
\begin{lstlisting}
function u0 = heatic(x)
T0=100; % temperature at left end, degrees C
if x==0
u0=T0;
else
u0=0;
end
end
\end{lstlisting}	

\vspace{5mm}
The full code for this example is 
\mylisting{heatConduction.m}

\begin{comment}
\newpage
\subsection{Solving a System of PDE}
\end{comment}

When executed, this script produces the following two figures
\begin{figure}[ht]
	\centering
	\includegraphics[angle=0,
	width=0.75\textwidth]{\exdir heatConduction_fig1}
	\label{fig:heatConduction_fig1}
\end{figure}
\begin{figure}[ht]
	\centering
	\includegraphics[angle=0,
	width=0.75\textwidth]{\exdir heatConduction_fig2}
	\label{fig:heatConduction_fig2}
\end{figure}

\subsubsection{Improving Performance with Vectorized Mode}
Execution time can often be substantially reduced by using the
vectorized option. The following changes to the script above are
required. The {\mycode heatpde} function is replaced with

\begin{lstlisting}
	function [c,f,s] = heatpde(x,t,u,DuDx)
	rho=8940; % material density
	cp=390; % material specific heat
	k=385; % material thermal conductivity 
	nx=length(x);
	c = rho*cp*ones(1,nx);
	f = k*DuDx;
	s = zeros(1,nx);
	end
\end{lstlisting}

And \pde is called as follows
\begin{lstlisting}
	options.Vectorized='on';
	u = pde1dm(m, @heatpde,@heatic,@heatbc,x,t,options);
\end{lstlisting}

\newpage
\subsection{Simple Coupling of PDE and ODE Equations}
The Numerical Algorithms Group (NAG) provides a function, \mycode{d03phf}, for solving coupled systems of PDE and ODE equations (reference \cite{nagD03}).
Their simple example coupling a single PDE with a single PDE will be shown here.

The single PDE, is

\begin{equation}\label{eq_simple_ode_pde_1}
	v^2\frac{\partial u}{\partial t} = \frac{\partial^2 u}{\partial x^2} +
	x v \frac{\partial v}{\partial t} \frac{\partial u}{\partial x}
\end{equation}

defined on the domain from $x=0$ to $x=1$

and the single ODE is

\begin{equation}\label{eq_simple_ode_pde_2}
	\frac{\partial v}{\partial t} = v u(1) +\frac{\partial u}{\partial x}(1) + t + 1
\end{equation}
where $u(1)$ is the solution at $x=1$ and ${\partial u}/{\partial x}(1)$ is the
x-derivative at $x=1$.

The left boundary condition at $x = 0$ is
\begin{equation}
	\frac{\partial u}{\partial x}=-v e^t
\end{equation}

The right boundary condition at $x = 1$ is
\begin{equation}
	\frac{\partial u}{\partial x}=-v\dot v
\end{equation}

An analytic solution to this problem is
\begin{eqnarray}
	u =& e^{(1-x)t} - 1 \\
		v=&t \\
\end{eqnarray}
The initial conditions on $u$ and $v$ in the \pde solution are computed
from this exact solution.

The code for this example is shown below
\mylisting{nagD03phfExample.m}

The code produces the following output and figures

\begin{lstlisting}
	Maximum error in ODE valriable=  2.01e-03
	Maximum error in PDE valriable=  1.02e-02
\end{lstlisting}

\begin{figure}[ht]
	\centering
	\includegraphics[angle=0,
	width=0.75\textwidth]{\exdir nagD03phfExample_fig1}
	\label{fig:nagD03phfExample_fig1}
\end{figure}

\begin{figure}[!ht]
	\centering
	\includegraphics[angle=0,
	width=0.75\textwidth]{\exdir nagD03phfExample_fig2}
	\label{fig:nagD03phfExample_fig2}
\end{figure}

\newpage
\subsection{Nonlinear Heat Equation with Periodic Boundary Conditions}
Here is another example showing the usefulness of being able to couple
ODE equations with the PDE equations.
For the solution to be periodic, the following two conditions must
hold

\begin{equation}\label{eq_periodic_sol}
 u(a)=u(b) 
\end{equation}
 
 and 
\begin{equation}\label{eq_periodic_dsol}
 \frac{\partial u}{\partial x}(a)=\frac{\partial u}{\partial x}(b)
\end{equation}
It is not possible
to apply such boundary conditions using the standard PDE boundary condition
mechanism. However reference \cite{post} shows how a periodic boundary
condition can be prescribed by adding an ODE to the system. This example
from reference \cite{post} is presented here. This example has the added
benefit that a simple analytical solution is available to compare with
the numerical results. 

We will enforce equation \eqref{eq_periodic_dsol} by defining an ODE
variable, $v$, such that

\begin{eqnarray}\label{eq_periodic_dsol_ode}
\frac{\partial u}{\partial x}(a)= v \\
\frac{\partial u}{\partial x}(b) = v
\end{eqnarray}

The ODE used to enforce equation \eqref{eq_periodic_sol} is simply
\begin{equation}\label{eq_periodic_sol_ode}
u(a)-u(b)=0
\end{equation}
Clearly this doesn't {\bf look} like an ODE because neither $v$ nor
$\partial v/\partial t$ is present. However it does satisfy the form
defined in equation \eqref{eq:ode};

The code for this example is shown below
\mylisting{schryer_ex4.m}

From Figure~\ref{fig:schryer_ex4_fig1} 
we see that the solution at the final time
compares well with the analytical solution. We also see that the
solutions at the right and left ends agree.
\begin{figure}
\centering
\includegraphics[angle=0,
width=0.75\textwidth]{\exdir schryer_ex4_fig1}
\caption{Solution at the final time as a function of $x$.}
\label{fig:schryer_ex4_fig1}
\end{figure}

Figure~\ref{fig:schryer_ex4_fig2} shows
that the solutions at the right and left ends are equal at all times
and that they agree with the analytical solution.
\begin{figure}
\centering
\includegraphics[angle=0,
width=0.75\textwidth]{\exdir schryer_ex4_fig2}
\caption{Solution at the two ends as a function of time.}
\label{fig:schryer_ex4_fig2}
\end{figure}
 
Figure~\ref{fig:schryer_ex4_fig3}
plots the solution of the ODE equation, $v$ as a function of time.
A simple differentiation of the analytical solution with respect to
$x$ shows that $\partial u/\partial x$ at the ends equals zero for all
times. The figure shows that the numerical solution is also very close
to zero for all times.
\begin{figure}
\centering
\includegraphics[angle=0,
width=0.75\textwidth]{\exdir schryer_ex4_fig3}
\caption{Solution of the ODE equation as a function of time.}
\label{fig:schryer_ex4_fig3}
\end{figure}

\newpage
\subsection{Dynamics of An Elastic Structural Beam}
This example shows how the structural dynamics of a beam can be analyzed with
\pde. The classical PDE for transient deflections of a beam is second order 
in time and fourth order in space. The form of PDE that \pde accepts is
first order in time and second order in space. One of the specific goals of
this example is to show how the beam equation can be converted to a form
acceptable to \pde.

The classical equation for the deflection of a beam is 

\begin{equation}\label{eq_beam_pde}
	\frac{\partial^2}{\partial x^2}(EI\frac{\partial^2 w}{\partial x^2})
	+ N\frac{\partial^2 w}{\partial x^2}
	+ \rho A\frac{\partial^2 w}{\partial t^2} = F
\end{equation}
The variables in this equation are:

\begin{tabular}{ll}
	$w$ & Transverse deflection of the beam \\
	$x$ & Coordinate along the beam axis \\
	$t$ & Time \\
	$E$ & Modulus of elasticity of the material \\
	$I$ & Moment of inertia of the beam cross section \\
	$N$ & Prescribed axial force in the beam \\
	$\rho$ & Density of the material \\
	$A$ & Cross sectional area of the beam \\
	$F$ & Distributed transverse loading on the beam \\
\end{tabular}

To convert this equation into a form acceptable to \pde, we first define
the following auxiliary variables

\begin{equation}\label{eq_beam_moment}
	\frac{\partial^2 w}{\partial x^2} = K
\end{equation}

and
\begin{equation}\label{eq_beam_velocity}
	\frac{\partial w}{\partial t} = \dot w = V
\end{equation}

With these two definitions equation \eqref{eq_beam_pde} can be rewritten as
this system of three PDE

\begin{equation}\label{eq_beam_system}
	\left\{\begin{array}{c}
		\dot w \\ \rho A \dot V \\ 0
	\end{array}\right\} =
	\left\{\begin{array}{c}
		0 \\ -EI\frac{\partial^2 K}{\partial x^2} -N\frac{\partial^2 w}{\partial x^2} \\
		-\frac{\partial^2 w}{\partial x^2}
	\end{array}\right\} +
	\left\{\begin{array}{c}
		V \\ F \\ K
	\end{array}\right\}
\end{equation}

The dependent variables in this system are

\begin{equation}\label{eq_beam_depvar}
	u = \left\{\begin{array}{c} w \\ V \\ K \end{array}\right\}
\end{equation}

\subsubsection{PDE Definition}
\pde requires that the user write a function with the following signature  to
define the system of PDE.
\begin{lstlisting}
	[c,f,s] = pdefun(x,t,u,dudx)
\end{lstlisting}
The details of the input and output arguments to this function are defined in
the \pde documentation.
A \mycode{pdefun} function defining equations 
\eqref{eq_beam_system} and \eqref{eq_beam_depvar} takes the following form
\begin{lstlisting}
	c = [1 rho*A 0]';
	f = [0 -E*I*DuDx(3)-N*DuDx(1) -DuDx(1)]';
	s = [u(2) F u(3)]';
\end{lstlisting}
\subsubsection{Boundary Conditions}
\pde requires that the user write a function with the following signature to
define the boundary conditions at the left and right ends of the spatial
region.
\begin{lstlisting}
	[pl,ql,pr,qr] = bcfun(xl,ul,xr,ur,t)
\end{lstlisting}
In the structural analysis of beams, three types of boundary conditions are frequently considered: free, simple support, and clamped. The two ends of the
beam may have any combination of these three basic types. In the code snippets
below, $u, p,$ and $q$ represent the value of the corresponding variable at
either the left or right end.

\bigskip
{\bf Free}
\begin{lstlisting}
	p = [0 0 u(3)]';
	q = [1 1 0]';
\end{lstlisting}

{\bf Simple Support}
\begin{lstlisting}
	p = [u(1) u(2) u(3)]';
	q = [0 0 0]';
\end{lstlisting}

{\bf Clamped}
\begin{lstlisting}
	p = [u(1) u(2) 0]';
	q = [0 0 1]';
\end{lstlisting}

\subsubsection{Free Vibration of a Simply Supported Beam}
As a simple example we will consider a beam that is initially displaced but
is otherwise unloaded and has simple support boundary conditions at both ends.
The initial displacement is chosen to be a half sin wave over the length of the
beam. This is the eigenvector for the lowest vibration frequency and this
allows for a particularly simple analytical solution. We will compare the
analytical solution with the solution obtained from \pde.

The complete listing of the \ml code for this example is shown below.
\mylisting{beamFreeVibration.m}

The results are shown in the following two figures. As can be seen, agreement
between the \pde and analytical solutions is very good.

\begin{figure}[!h]
	\centering
	\includegraphics[angle=0,
	width=0.75\textwidth]{beam_free_vibr_1.png}
	\label{fig:beam_free_vibr_1}
\end{figure} 
\begin{figure}[!h]
	\centering
	\includegraphics[angle=0,
	width=0.75\textwidth]{beam_free_vibr_2.png}
	\label{fig:beam_free_vibr_2}
\end{figure} 

\newpage
\subsection{Melting of Ice}
This example shows the melting of a semi-infinite ice slab. We have chosen the somewhat
unrealistic semi-infinite case because it has a simple analytical solution that we will
compare with. Most of what is learned in this example can be applied to more realistic
solidification and melting problems and also problems where moving coordinate systems are 
advantageous. 

Problems of this type are often referred to as Stefan problems named after the physicist
who studied melting and freezing in the 1800's. They are characterized by a moving ``front''
that separates one material phase from the other. They are discussed in many publications
including \cite{meltingFreezing}.

The specific problem we will consider here is a slab of ice at zero degrees-C
with a temperature of 25 degrees-C suddenly applied to the left ($x=0$) face. The
extent of the block in the other directions is assumed to be large. This allows the
use of a one-dimensional model of heat transfer and implies that the block never fully
melts during the time of the simulation. We model only the liquid phase that initially 
has length zero and lengthens as heat is transferred from the left edge into the ice. Such 
a model is referred to as a one-phase Stefan model.

A key parameter of such a model is the evolving location of the liquid-solid interface. 
This is given by the Stefan equation, an ordinary differential equation that relates the
heat flux in the ice of the velocity of the interface location:



The complete listing of the \ml code for this example is shown below.
\mylisting{meltingIce.m}

\newpage
\appendix
\section{Appendix: Comparison Between \pde and \ml \pdepe} \label{sec:pdepe} 
The types of PDE solved by \pde and \ml  {\tt pdepe} are similar. The input to \pde is
also similar to that of {\tt pdepe} and, in many cases, the  {\tt pdepe} input for a
problem can be used without changes in \pde. However, there are some significant differences
between the two PDE solvers and this section describes those. It is primarily written for users
who have a reasonable amount of experience with {\tt pdepe}.

\begin{enumerate}
\item
\pdepe has interpolation functions specially designed to provide accurate solutions near $x=0$
in problems with cylindrical ($m=1$) and spherical ($m=2$) coordinate systems. \pde can solve
the PDE for such problems but the solution will generally be less accurate near $x=0$.
\item
\pdepe allows the user to specify {\tt Events} to be handled during the solution. \pde does not currently support events.
\item
\pdepe relies on the \ml {\tt ode15s} ODE solver to solve initial value problem. \pde uses the {\tt ode15i} solver.
\item 
The Octave version of {\tt ode15i} cannot solve ODE with complex matrices so this limits
\pde, when used with Octave, to only real PDE.
\end{enumerate}

% references
\begin{thebibliography}{99}
\bibitem{post} N. L. Schryer, POST- A Package for Solving Partial
Differential Equations in One Space Variable, AT\&T Bell Laboratories,
August 30, 1984.
\bibitem{howard}
\href{https://www.math.tamu.edu/~phoward/m401/pdemat.pdf}
{Partial Differential Equations in MATLAB 7.0}
\bibitem{nagD03}
\href{https://www.nag.co.uk/numeric/fl/manual/pdf/D03/d03phf.pdf}
{NAG Fortran Library Routine Document D03PHF/D03PHA}.
\bibitem{meltingFreezing}
 Vasilios Alexiades and  Alan D. Solomon,
 {\it Mathematical Modeling Of Melting And Freezing Processes},
 (Taylor \& Francis), 1993.
\end{thebibliography}

\end{document}